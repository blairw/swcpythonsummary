\section{Repeating Actions with Loops}
\begin{multicols*}{2}

Consider the example from the previous section in which we created \mintinline{python}{axes1}, \mintinline{python}{axes2} and \mintinline{python}{axes3} and placed them into a \mintinline{python}{matplotlib} figure. If you look at the code for \mintinline{python}{axes1}, \mintinline{python}{axes2} and \mintinline{python}{axes3} (lines 3 to 5, 7 to 9, and 11 to 13), you'll notice they are all quite similar. The only things that differ for each of these is:

\begin{myitemize}
    \item The third parameter for the \mintinline{python}{fig.add_subplot()} function;
    \item the title of the graph; and,
    \item the data source for the graph.
\end{myitemize}

Everything else about the graphs was the same. This is a great example of when we could use a \textbf{for-loop}. The code sample below \textbf{refactors} \textit{(= rewriting code to make it more elegant, while retaining the same outcome of the code)} the previous code sample to use a for-loop.

\vspace{-4mm}
\begin{minted}[xleftmargin=6mm,frame=lines,framesep=2mm,linenos,fontsize=\small]{python}
fig = matplotlib.pyplot.figure(figsize=(10.0, 3.0))

figure_data = [
    {'title': 'Daily mean',      'numbers': numpy.mean(data, axis=0)}
    , {'title': 'Daily maximum', 'numbers': numpy.max(data, axis=0)}
    , {'title': 'Daily minimum', 'numbers': numpy.min(data, axis=0)}
]

for seqno in range(0, 3):
    this_axes = fig.add_subplot(1, 3, seqno + 1)
    this_axes.set_xlabel('day')
    this_axes.set_ylabel('inflammation')
    this_axes.set_title(figure_data[seqno]['title'])
    this_axes.plot(figure_data[seqno]['numbers'])

fig.tight_layout()
matplotlib.pyplot.savefig('inflammation_figure.pdf')
\end{minted}

\par
A few things to note here:

\begin{myitemize}
    \item Line 1 is the same as it was before.
    \item On lines 3 to 7, we now have a data structure called \mintinline{python}{figure_data}. This is an array with two elements, each of which is in the format \mintinline{python}{{ 'k1': v1, 'k2': v2 }}. This is known as a \textbf{dictionary}. It allows us to store \textbf{values} (e.g. \textit{k1, k2}) for specified \textbf{keys} (e.g. \textit{v1, v2}). So, for example, the value stored at \mintinline{python}{figure_data[0]['title']} is the string \textit{``Daily mean''}.
    \item On lines 9 to 14, we now have a control structure called a \textbf{for-loop}. This means that everything ``inside'' the loop (lines 10-14) is executed in each iteration of the loop. \textbf{In Python, we indicate what is ``inside'' the loop using indentation, which is why lines 10-14 are indented.} This is very important if you are coming from a language like Java where indentation is not as consequential.
    \item On line 9, we define the for-loop using \mintinline{python}{range(0, 3)}. This is a function that generates an array from 0 (inclusive) to 3 (exclusive), i.e., \textit{\{0, 1, 2\}}. The loop iterates across each of these elements, with each iteration storing the element as \mintinline{python}{seqno}. This conveniently means that while we traverse \mintinline{python}{range(0, 3)}, we also traverse the elements of \mintinline{python}{figure_data}, which becomes very useful on lines 13 and 14. It also means, technically, we could have used \mintinline{python}{range(0, len(figure_data))} instead of \mintinline{python}{range(0, 3)}.
    \item However, because \mintinline{python}{fig.add_subplot()} expects \textit{\{1, 2, 3\}} for its third parameter (i.e. 1-indexed instead of 0-indexed), we have the expression \mintinline{python}{seqno + 1} on line 10.
    \item On line 9, we could have alternatively used the expression \mintinline{python}{for seqno, this_value in enumerate(figure_data)}. This would allow us to replace \mintinline{python}{figure_data[seqno]} with simply \mintinline{python}{this_value} on lines 13 and 14. This is similar to a \textbf{for-each loop} (which you may be familiar with if you have a Java background).
\end{myitemize}

You should also know that you can treat a string as an array of one-character strings, and use a for-loop on it likewise:

\vspace{-4mm}
\begin{minted}[xleftmargin=6mm,frame=lines,framesep=2mm,linenos,fontsize=\small]{python}
word = 'oxygen'
for char in word:
    print(char)
\end{minted}

\end{multicols*}