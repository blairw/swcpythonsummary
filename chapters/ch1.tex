\section{Python Fundamentals}
\begin{multicols*}{2}
\subsection{Variables}
\par
To start with, we can think of Python as a calculator so you can do a calculation like ``3 + 5 × 4'':

\vspace{-4mm}
\begin{minted}[xleftmargin=6mm,frame=lines,framesep=2mm,linenos]{python}
>>> 3 + 5 * 4
23
\end{minted}

\par
The above is what it looks like in an interactive Python shell (\mintinline{zsh}{python3} at a bash/zsh shell). However, if you are running a Python \textit{script} (e.g. \mintinline{zsh}{python3 myscript.py}), you will not see anything. In that latter case, you will need to type into your Python script:

\vspace{-4mm}
\begin{minted}[xleftmargin=6mm,frame=lines,framesep=2mm,linenos]{python}
print(3 + 5 * 4)
\end{minted}

\par
Like any good `scientific' calculator, you can also store the results of calculations into variables like how you can do \texttt{Ans $\rightarrow$ A}. However, unlike your typical high school scientific calculator, Python gives you much more flexibility in naming your variables. Therefore, you should give your variables meaningful names. \mintinline{python}{A} would actually be a very bad name for a variable in most cases. Your variables should be named according to the business context, e.g. if we were writing some Python script about the Consumer Price Index (CPI):

\vspace{-4mm}
\begin{minted}[xleftmargin=6mm,frame=lines,framesep=2mm,linenos]{python}
cpi_june_2019 = 0.6
print(cpi_june_2019)
\end{minted}

\subsection{Data Types}
You will notice in the above we first worked with \textbf{integers} ($\mathbb{Z}$, e.g. 1, 2, 3) and then there was an example with a \textbf{rational number} ($\mathbb{Q}$, e.g. 0.6, $\frac{1}{3}$). Rational numbers are stored in Python as so-called ``floating-point'' numbers or ``floats''. Floats can sometimes behave strangely:

\vspace{-4mm}
\begin{minted}[xleftmargin=6mm,frame=lines,framesep=2mm,linenos]{python}
>>> 1.1 + 1.3
2.4000000000000004
\end{minted}

To cope with this, we can use \textbf{libraries}. A library is some additional component of Python that gives us access to additional programming functionality, e.g.:

\vspace{-4mm}
\begin{minted}[xleftmargin=6mm,frame=lines,framesep=2mm,linenos]{python}
>>> from fractions import Fraction
>>> from decimal import Decimal
>>>
>>> Decimal(1.1) + Decimal(1.3)
Decimal('2.400000000000000133226762955')
>>> # this ^ is still the wrong answer
>>> # we should enclose the numbers in quotes
>>> 
>>> Decimal('1.1') + Decimal('1.3')
Decimal('2.4')
>>> # correct answer! :)
>>>
>>> Fraction(1,3) + Fraction(1,3)
Fraction(2, 3)
>>> print(Fraction(1,3) + Fraction(1,3))
2/3
>>> print(float(Fraction(1,3) + Fraction(1,3)))
0.6666666666666666
\end{minted}

You may notice in the above that I have used the ``hash'' symbol to make \textbf{comments}. This is very useful for explaining your code.

\subsection{Additional Tips}
\begin{myitemize}
\item If you want to, you can assign multiple variables at the same time, e.g. \mintinline{python}{mass, age = 25.5, 20}. Only do this if it makes your code more readable.
\item In additional to the numerical data types discussed above, Python also has the data type for text strings (e.g. the classic: \mintinline{python}{print('Hello world')}). You can use either single quotes or double quotes. This can be useful in some situations, e.g.  \mintinline{python}{print('In French: "Bonjour tout le monde"')}.
\item Variable names can include letters, digits and underscores. However, they cannot \textit{begin} with a digit. For example, you cannot create a variable called  \mintinline{python}{12th_person}.
\end{myitemize}


\end{multicols*}